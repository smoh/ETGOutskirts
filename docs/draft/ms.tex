\documentclass{aastex}

% packages
\usepackage{natbib}
\usepackage{amsmath}


% new commands go here
\newcommand{\HI}{\ion{H}{1}}
\newcommand{\Sersic}{S\'ersic}

% my todo command
\usepackage{xcolor}
\newcommand\todo[1]{\textcolor{red}{#1}}
% \renewcommand\todo[1]{}  % hide todos

\begin{document}

\title{Outskirts of Elliptical Galaxies}
\author{authors}


\begin{abstract}
    abstract    
\end{abstract}    


\section{Introduction}
The formation of elliptical galaxies is one of the important open questions on the path way to
understanding galaxy formation and evolution in $\Lambda$CDM universe -- why?

In the recent years, many studies have reported and confirmed the size growth of ETGs.

\section{Data \& Sample}

\subsection{Data}
We use the SDSS ``corrected frame'' images. These images have been sky-subtracted using an improved
method of estimating global sky by fitting a smooth spline to the previous PHOTO sky estimates
\citep[see][for details]{blanton2011}.

For the fitting, we use cutouts of the frame images. Although the global sky estimates have been subtracted,
a systematic residual background of typically $26-28$ mag/arcsec$^2$ remains.
To include enough background pixels for estimating this residual sky level, we make use of the single \Sersic{}
parameters provided by the NSA catalog,
and set the optimal cutout size to be twice the radius at which the surface
brightness of the galaxy is 30~mag/arcsec$^2$.
The typical optimal cutout sizes range from \todo{some} arcsec ($40-80$ petrosian radius).
We require that more than 70\% of the square area set by the optimal cutout size be covered in a single frame image,
and exclude galaxies that do not meet this criterion.
About \todo{20\%} of the sample is excluded.

We use NSA pimages to mask out neighbors in each cutout image. In many cases, these masks treat nearby bright
stars as connected to the galaxy.
In addition, we have no hopes of accurately fitting the surface brightness for galaxies with overlapping neighbor
near the very center.
We first do a single \Sersic{} fitting on images with the NSA masks applied, and use the residual images to create
additional masking, and reject those with close neighbors.
To mask out the remaining neighbors, 

We examine whether our selection for fittable images introduces biases in the final eample.

\section{Fitting}
\subsection{Size from the Fundamental Plane}

\begin{eqnarray}
       \mu_0 &= 2.5 + \log(2 \pi r^2) - 10 \log(1+z)
   \end{eqnarray}   

The effective circular size $R_0$ in kpc is related to $r$ by
the angular diameter distance $D_A$ (kpc/arcsec):

\begin{align}
    r &= R_0 / D_A \\
    \log R_0 &= a \log \sigma - \frac{b}{2.5} \mu_0 + c \\
    \log R_0 &= \frac{1}{1 + 2 b} \left[ a\log \sigma
        - \frac{b}{2.5} \left(m + 2.5\log \frac{2\pi}{D_A^2} - 10 \log (1+z)\right)
        + c \right]\\
\end{align}
With $b \sim -0.75$, $\log R_0 \propto -0.6 m$


The \Sersic{} profile

\begin{align}
    I(R) &= I_e \exp \left[ -b_n \left( \frac{R}{R_e} \right)^{1/n} - 1 \right]
\end{align}
We take analytic form of $b_n$ from Lima-Neto et al. 1999
\begin{align}
    b_n &= n \exp\left(0.6950 - \frac{0.1789}{n}\right)
\end{align}



\end{document}
