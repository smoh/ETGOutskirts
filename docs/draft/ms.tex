\documentclass[iop]{emulateapj}

% packages
\usepackage{natbib}
\usepackage{amsmath}
\usepackage{graphicx}
% \usepackage{subcaption}

% new commands go here
\newcommand{\HI}{\ion{H}{1}}
\newcommand{\Sersic}{S\'ersic}
\newcommand{\chisq}{\ensuremath{\chi^2}}


% my todo command
\usepackage{xcolor}
\newcommand\todo[1]{\textcolor{red}{#1}}
% \renewcommand\todo[1]{}  % hide todos

\graphicspath{{figures/}}

\begin{document}

\title{Outskirts of Elliptical Galaxies}
\author{authors}


\begin{abstract}
    abstract    
\end{abstract}    


\section{Introduction}

In the recent years, many studies have reported and confirmed the size growth
of ETGs.

\section{Data \& Sample}

\subsection{Data}
We use the SDSS corrected frame images. These images have been
sky-subtracted using an improved method of estimating global sky by fitting
a smooth spline to the previous PHOTO sky estimates
\citep[see][for details]{blanton2011}.

For the fitting, we use cutouts of the frame images.
Although the global sky estimates have been subtracted, a systematic residual
background of typically $26-28$ mag/arcsec$^2$ remains.
To include enough background pixels for estimating this residual sky level,
we make use of the single \Sersic{} parameters provided by the NSA catalog,
and set the optimal cutout size to be twice the radius at which the surface
brightness of the galaxy is 30~mag/arcsec$^2$.
The typical optimal cutout sizes range from
\todo{some} arcsec ($40-80$ petrosian radius).
We require that more than 70\% of the square area set by the optimal cutout
size be covered in a single frame image,
and exclude galaxies that do not meet this criterion.
About \todo{20\%} of the sample is excluded.

We use NSA pimages to mask out neighbors in each cutout image.
In many cases, these masks treat nearby bright stars as connected to the galaxy.
In addition, we have no hopes of accurately fitting the surface brightness
for galaxies with overlapping neighbor near the very center.
We first do a single \Sersic{} fitting on images with the NSA masks applied,
and use the residual images to create additional masking,
and reject those with close neighbors.
To mask out the remaining neighbors, 

We examine whether our selection for images introduces biases
in the final sample.

\begin{figure*}[]
    \begin{center}
        \includegraphics[width=.32\textwidth]{distZ_sample}
        \includegraphics[width=.32\textwidth]{distPetrorMag_sample}
        \includegraphics[width=.32\textwidth]{distSigma_sample}
    \end{center}
    \caption{Distribution of redshift, Petrosian r magnitude, and velocity
        dispersion before and after excluding bad images.
        Bottom panels show the fraction of excluded galaxies in each bin.}
    \label{fig:SampleImageBias}
\end{figure*}

\begin{figure*}[]
    \begin{center}
        \includegraphics[width=0.95\textwidth]{gridParams}
    \end{center}
    \caption{Correlation between key parameters of deVExp model. Red points
        highlight double component galaxies.}
    \label{fig:CornerPlot}
\end{figure*}

\begin{figure*}[]
    \begin{tabular}{cc}
        \includegraphics[width=.45\textwidth]{distBT} &
        \includegraphics[width=.45\textwidth]{distSersicN}\\
        \includegraphics[width=.45\textwidth]{distSigma} &
        \includegraphics[width=.45\textwidth]{distReRatio}
    \end{tabular}
    \caption{Distribution of $B/T$, \Sersic index, $\sigma$ and
        $R_{e,1}/R_{e,4}$ for single and double component galaxies.}
    \label{fig:DoubleCompParams}
\end{figure*}

\begin{figure*}[]
    \begin{center}
        \includegraphics[width=0.9\textwidth]{componentFP}
    \end{center}
    \caption{The Fundamental Plane relation for deV (red) and Exp (blue)
        component of deVExp model. The two figures are identical except
        highlighted component. Larger symbols are for double component
        galaxies.}
    \label{fig:ComponentFP}
\end{figure*}

\begin{figure}[]
    \begin{center}
        \includegraphics[width=.45\textwidth]{diff_chi2}
    \end{center}
    \caption{$\triangle\chisq$ for single and double (red) component galaxies.}
    \label{fig:DiffChisq}
\end{figure}

\begin{figure*}[]
    \begin{center}
        \includegraphics[width=0.45\textwidth]{Re_q_exp}
        \includegraphics[width=0.45\textwidth]{Re_qratio}
    \end{center}
    \caption{Effective radius of Exp component vs. its axis ratio}
    \label{fig:Req}
\end{figure*}

\section{Fitting}
\subsection{Size from the Fundamental Plane}



\begin{eqnarray}
       \mu_0 &= 2.5 + \log(2 \pi r^2) - 10 \log(1+z)
\end{eqnarray}

The effective circular size $R_0$ in kpc is related to $r$ by
the angular diameter distance $D_A$ (kpc/arcsec):

\begin{equation}
    r = R_0 / D_A \\
\end{equation}

\begin{equation}
    \log R_0 = a \log \sigma - \frac{b}{2.5} \mu_0 + c
\end{equation}

\begin{equation}
    \log R_0 = \frac{1}{1 + 2 b} \left[ a\log \sigma
        - \frac{b}{2.5} \left(m + 2.5\log \frac{2\pi}{D_A^2} - 10 \log (1+z)\right)
        + c \right]
\end{equation}
With $b \sim -0.75$, $\log R_0 \propto -0.6 m$


The \Sersic{} profile

\begin{align}
    I(R) &= I_e \exp \left[ -b_n \left( \frac{R}{R_e} \right)^{1/n} - 1 \right]
\end{align}
We take analytic form of $b_n$ from Lima-Neto et al. 1999
\begin{align}
    b_n &= n \exp\left(0.6950 - \frac{0.1789}{n}\right)
\end{align}



\end{document}
